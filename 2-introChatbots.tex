\chapter{Introducción a los Chatbots}

En este capítulo se tratará de abordar uno de los conceptos fundamentales en los que se basa el trabajo realizado. No es otro que los Chatbots. En el Apartado \ref{sec:DefinicionChatbot} se tratará de describir qué es un Chatbot, en qué contextos se utiliza y se justificará la utilidad de los mismos en el contexto de las consultas a hojas de cálculo. En el Apartado \ref{sec:PlataformasChatbot} se explicará las actuales plataformas para el desarrollo de Chatbots, sus características y la solución que se ha adoptado en este trabajo.

\section{Definición y contexto de uso}
\label{sec:DefinicionChatbot}

Existen múltiples definiciones para describir qué es un Chatbot, agente conversacional, o simplemente bot. Un chatbot se puede describir como un software para automatizar infinidad de tareas que actualmente desarrollan los usuarios por si mismos, como reservar un restaurante para cenar, añadir un evento al calendario u obtener información \cite{Wagner2016}. También se les describe como software con características de inteligencia artificial que pueden hacer cualquier cosa, enseñar, jugar, buscar, recordar, conectar, integrar con otros servicios,... \footnote{¿Qué es un chatbot de telegram? \url{https://telegram.org/blog/bot-revolution}}.

Lo que las definiciones dejan claro es que un bot conversacional tiene un propósito especifico pero que en lineas generales es un software que debe de lidiar con o resolver tareas cotidianas del usuario.

Cualquiera puede pensar que los bots son un concepto nuevo dentro de las tecnologías de información. Sin embargo es una idea que lleva desde los comienzos de la informática (uno de los pioneros fue el proyecto ELIZA\footnote{Proyecto ELIZA, un chatbot que simulaba a una psicóloga: \url{https://en.wikipedia.org/wiki/ELIZA}}), aunque en los últimos años está en crecimiento \cite{Ferrara2016}. Esto es debido al aumento de las redes sociales y de los dispositivos interconectados dentro de lo que se conoce cómo el Internet de las Cosas \footnote{Utilización de chatbots como interfaces para el Internet de las cosas: \url{https://iot.telefonica.com/blog/using-smart-chatbots-as-an-iot-interface}}.

Los chatbots proporcionan una interfaz de comunicación en la que se reduce el coste frente a la interacción humana que tenían que proporcionar las empresas \cite{Dans2016}. De igual manera también porque esta interacción ejerce menor presión en el usuario que quiere realizar consultas. Un bot está disponible para atender consultas 24 horas al día los 7 días de la semana, y puede atender simultáneamente consultas de múltiples usuarios, a diferencia de los humanos.

A pesar de citar 


\section{Plataformas para desarrollo de agentes conversacionales}
\label{sec:PlataformasChatbot}