\chapter{\introduction}

% TODO Revision and fix

La inclusión del Smartphone como herramienta de comunicación y búsqueda de información se ha extendido superando a los sistemas de cómputo tradicionales como el PC o los portátiles. El Smartphone dispone actualmente una capacidad de trabajo similar a los PC con la ventaja de la movilidad que ofrece. En la actualidad, con un Smartphone se pueden realizar la mayoría de tareas cotidianas que un usuario puede requerir, como leer el correo electrónico, comunicarse con sus seres queridos, consultar información en la web o realizar compras online.

Sin embargo, a pesar de que se puedan realizar tareas complejas, sus limitaciones provoca que algunas tareas puedan ser realmente tediosas o imposibles de realizar. Un ejemplo claro es la consulta de información de datos en hojas de cálculo. En la actualidad el uso de hojas de cálculo como Microsoft Excel o Google Spreadsheet es una de las herramientas más utilizadas en el manejo de información, en el ámbito empresarial, pero también a nivel personal. La potencia y versatilidad que ofrece es de sobra conocida, de ahí que exista gran cantidad de hojas de cálculo para el almacenamiento de datos. Actualmente 1 de cada 7 habitantes en el mundo utilizan alguna herramienta de hojas de cálculo, para almacenar información, pero también para consultarla. %TODO Buscar referencia

Como se ha comentado previamente, el uso del smartphone ha proliferado en los últimos años, donde su característica principal es la movilidad que ofrece frente a los PC o portátiles tradicionales. Para ofrecer esta movilidad una de las características hay características








Un ejemplo claro es la navegación por sitios web. A pesar de que el uso de las tecnologías web permite que un mismo sitio web se pueda consultar desde cualquier plataforma, la realidad es que hay muchísimos sitios web que aún no están preparados para su navegación desde un Smartphone. Debido al reducido tamaño y la interacción táctil de los Smartphone, la experiencia de usuario suele ser bastante pobre en sitios web tradicionales.

Es por ello que muchos servicios se ofrecen mediante el uso de aplicaciones orientadas al Smartphone, como son las que se pueden encontrar en las tiendas de aplicaciones de Google Play, Apple Store o Windows Store. Entre las más populares se encuentran las aplicaciones de mensajería instantánea, redes sociales o videojuegos, aunque existe un abanico de posibilidades enorme. Aplicaciones de mensajería instantánea como Whatsapp o Telegram han superado los 1000 millones de descargas {buscar referencia}.

Parte del éxito de la mensajería instantánea se basa en algunas de las características de estas aplicaciones. La principal es la necesidad/posibilidad de conversar de manera asíncrona entre personas en cualquier momento y en cualquier lugar. Este tipo de herramientas están sustituyendo a las tradicionales llamadas de voz (síncronas) o a los correos electrónicos (complejos para mantener una conversación).

Aunque las propias características del Smartphone hacen que la mensajería instantánea basada en texto sea una de las alternativas más recurridas. El tamaño de la pantalla de los Smartphone impide que en ella se puedan visualizar datos de gran volumen como tablas o gráficas muy complejas. Tampoco son herramientas muy adecuadas para dibujar o realizar escritura manual, en favor del teclado virtual, que es un método de entrada fiable y sencilla de usar. La conectividad de la que se dispone es generalmente WiFi o redes móviles como 3G o 4G, habitualmente asociadas a una tarifa de datos bastante restrictiva o limitada.

Asimismo, el entorno en el que se encuentra el usuario le impide algunas interacciones. Por ejemplo en las bibliotecas, o medios de transporte donde suele ser habitual que se mantenga cierto silencio, el uso de comandos de voz es cada vez más sustituido por el chat.

Dadas estas características, se justifica el aumento del uso de las aplicaciones de mensajería instantánea como método de comunicación entre seres humanos, pero también es interesante la comunicación con servicios o maquinas, paradigma conocido como Human-To-Machine communication. La comunicación entre humanos y máquinas existe desde los años 70 {Buscar referencia}. De aquí surge el concepto de chatbot o agentes conversacionales.

Un bot o chatbot es un sistema o aplicación que es capaz de interpretar comandos textuales de un ser humano y ofrecer una respuesta, de tal manera que permite al usuario entablar una conversación. A estos sistemas se les conoce como Interfaces de Conversación para Usuarios (Conversational User Interfaces, CUI).

Se podría considerar entre los primeros CUIs a los interpretes de la línea de comandos que datan del 1979 (Circa). Este sistema permitía al usuario introducir comandos que se interpretaban y ejecutaban. 

Un chatbot como se ha mencionado, tiene dos cometidos principales, interpretar comandos textuales y ser capaz de obtener una solución aceptable para el usuario, ya sea por medio de ejecutar una tarea concreta o facilitar una respuesta concisa al usuario ante una cuestión. Para resolver estas tareas se aúnan diferentes técnicas o tecnologías.

Para interpretar los comandos de usuario se utilizan técnicas de procesamiento del lenguaje o análisis de sentimientos. Es necesario ser capaces de interpretar con claridad el objetivo de usuario, si no será imposible ofrecerle una respuesta acorde a su petición. Por ejemplo, si son las 6 de la tarde y el usuario pide “avísame dentro de 15 minutos” o pide “establece una alarma para las seis y cuarto”, en realidad está pidiendo la misma tarea, que es que suene una alarma acústica a las 18:15. Sin embargo, la manera de pedirlo es completamente diferente y el chatbot debe ser capaz de interpretar el objetivo inequívocamente.

De igual manera, es imposible proporcionar una buena solución si el sistema no dispone de los recursos para ejecutarla. Por ejemplo si le preguntásemos a un chatbot por cual es el camino más corto para llegar a la universidad, pero no dispone de un servicio de mapas actualizado, no podrá proporcionarlo o lo proporcionará erróneamente.

Es por ello que un chatbot debe de lidiar con estas problemáticas, con tal de ofrecer una buena experiencia del usuario.

Como se ha mencionado previamente la movilidad que ofrece el Smartphone permite una comunicación a nivel global en cualquier momento y en cualquier sitio. Sin embargo, el disponer de una herramienta tan potente como esta ha permitido abrir el abanico de nuevas necesidades. Por ejemplo, ahora los usuarios les interesa buscar horarios de trenes, el tiempo que hará mañana o el calendario de su equipo favorito.

Esto supone un problema un problema muchas veces de que hay que acceder a un sitio web y realizar ciertos pasos antes de obtener la información que se desea. Otras veces simplemente el problema es que el usuario no sabe dónde puede encontrar esa información.

Una posible solución es el uso de los antes mencionados chatbots. El definir un chatbot que sea capaz de resolver preguntas relacionadas con horarios de trenes, del tiempo o de futbol, permitiría resolver esas problemáticas con una sencilla pregunta.

Sin embargo a creación de bots, con la compleja cantidad de fraemeworks, servicios de mensaje instantáneo o lenguajes de programación que existen, solo permite que usuarios con altos conocimientos y/o con tiempo suficiente puedan crear bots que solucionen problemas concretos.

De aquí surge la problemática a resolver en este proyecto, cómo se le puede abstraer al desarrollador estos aspectos de programación complejos para permitirle realizar un bot de manera sencilla. Para ello, tal y como se ha profundizado antes, es importante diferenciar los dos aspectos, la conversación con el usuario y la recolección de datos o conocimiento con el que se debe responder a las cuestiones planteadas por el usuario. Dada la complejidad de recoger datos no estructurados, se ha optado por trabajar con datos estructurados de forma tabular, hojas de cálculo. Estos datos serán consultados por el motor del chatbot en base a las peticiones de entrada salida que se definan.	

Supongamos que un usuario necesita conocer el horario de bus y el de tren ya que hace transbordo para ir a su trabajo, quizá también le puede interesar conocer el saldo de su bono de transporte. Habitualmente el conocimiento de estos tres aspectos están repartidos en diferentes servicios, sitios web o aplicaciones. Resulta innecesario disponer de tres aplicaciones o tener que acceder a tres sitios web diferentes cada día. Asimismo podría darse el caso de que un día por culpa del trafico se retrasase el 
