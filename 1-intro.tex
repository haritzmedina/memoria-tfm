\chapter{\introduction}

% TODO Revision and fix

La inclusión del Smartphone se ha extendido hasta el punto de ser una herramienta indispensable en el día a día, tanto para comunicación \cite{Montag2015}, como para la búsqueda de información \cite{Wang2016}. El uso del Smartphone en estos aspectos está superando a los sistemas de cómputo tradicionales como el PC o los portátiles. El Smartphone dispone actualmente una capacidad de trabajo similar a los PC, con la ventaja de la movilidad que ofrece. En la actualidad, con un Smartphone se pueden realizar la mayoría de tareas cotidianas que un usuario puede requerir, como leer el correo electrónico, comunicarse con sus seres queridos, consultar información en la web o realizar compras online.

Como se ha comentado previamente, el uso del Smartphone ha proliferado en los últimos años, donde su característica principal es la movilidad que ofrece frente a los PC o portátiles tradicionales. Para ofrecer esta movilidad una de las características más afectada es la del tamaño del dispositivo. Se ha pasado de las pantallas mayores de 15 pulgadas a dispositivos que llegan a un máximo de 7"  (los conocidos como phablets\footnote{Los phablet son dispositivos móviles denominados de esta manera por comprenderse en un tamaño mayor que los smartphones (hasta 5") y menor que los tablets (a partir de 7"): \url{https://en.wikipedia.org/wiki/Phablet}}).


Sin embargo, a pesar de que se puedan realizar tareas complejas, sus limitaciones provoca que algunas tareas puedan ser realmente tediosas, o incluso, imposibles de realizar. Un ejemplo claro es la consulta de información de datos en hojas de cálculo. En la actualidad el uso de hojas de cálculo como Microsoft Excel o Google Spreadsheet es una de las herramientas más utilizadas en el manejo de información, en el ámbito empresarial, pero también a nivel personal. La potencia y versatilidad que ofrece es de sobra conocida, de ahí que exista gran cantidad de hojas de cálculo para el almacenamiento de datos. Actualmente 1 de cada 7 habitantes en el mundo utilizan alguna herramienta de hojas de cálculo, para almacenar información, pero también para consultarla. %TODO Buscar referencia

