\chapter{\introduction}

% TODO Revision and fix

La inclusión del Smartphone se ha extendido hasta el punto de ser una herramienta indispensable en el día a día, tanto para comunicación \cite{Montag2015}, como para la búsqueda de información \cite{Wang2016}. El uso del Smartphone en estos aspectos está superando a los sistemas de cómputo tradicionales como el PC o los portátiles. El Smartphone dispone actualmente una capacidad de trabajo similar a los PC, con la ventaja de la movilidad que ofrece. En la actualidad, con un Smartphone se pueden realizar la mayoría de tareas cotidianas que un usuario puede requerir, como leer el correo electrónico, comunicarse con sus seres queridos, consultar información en la web o realizar compras online.

Como se ha comentado previamente, el uso del Smartphone ha proliferado en los últimos años, donde su característica principal es la movilidad que ofrece frente a los PC o portátiles tradicionales. A pesar de las ventajas que ofrece, el uso de dispositivos móviles acarrea la inclusión de nuevos factores que pueden dificultar su usabilidad \cite{zhang2005challenges}:
\begin{itemize}
	\item \textbf{Contexto móvil}: el usuario puede cambiar su localización durante su uso. Asimismo, esto puede incluir interacción con su entorno, personas, objetos o el ambiente más próximo.
	\item \textbf{Conectividad}: la conectividad de los dispositivos móviles es lenta y poco fiable.
	\item \textbf{Tamaño de la pantalla}: la cantidad de información que se puede mostrar se reduce debido al tamaño de la pantalla que llega a un máximo de 7`` (los conocidos como phablets\footnote{Los phablet son dispositivos móviles denominados de esta manera por comprenderse en un tamaño mayor que los smartphones (hasta 5") y menor que los tablets (a partir de 7"): \url{https://en.wikipedia.org/wiki/Phablet}}).
	
\end{itemize}

Para ofrecer esta movilidad una de las características más afectada es la del tamaño del dispositivo. Se ha pasado de las pantallas mayores de 15 pulgadas a dispositivos que llegan a un máximo de 7"  (los conocidos como phablets\footnote{Los phablet son dispositivos móviles denominados de esta manera por comprenderse en un tamaño mayor que los smartphones (hasta 5") y menor que los tablets (a partir de 7"): \url{https://en.wikipedia.org/wiki/Phablet}}).

Sin embargo, a pesar de que se puedan realizar tareas complejas, sus limitaciones provoca que algunas tareas puedan ser realmente tediosas, o incluso, imposibles de realizar.

Un ejemplo claro es la consulta de información de datos en hojas de cálculo. En la actualidad el uso de hojas de cálculo como Microsoft Excel o Google Spreadsheet es una de las herramientas más utilizadas en el manejo de información, en el ámbito empresarial, pero también a nivel personal. La potencia y versatilidad que ofrece es de sobra conocida, de ahí que exista gran cantidad de hojas de cálculo para el almacenamiento de datos. Actualmente cerca de 750 millones de usuarios utilizan Microsoft Excel para presentar y analizar datos \cite{Investitech2015}.

Por lo tanto, este trabajo basa en la premisa de que es necesario el uso de hojas de cálculo en un contexto móvil o donde el usuario no tiene acceso a un ordenador, y en la premisa de que el acceso con el dispositivo móvil a las hojas de cálculo es complejo.

Un estudio reciente ha demostrado que el 79\% de los participantes (seleccionados a traves del European Spreadsheet Risk Interest Group\footnote{Sitio web de EuSpRiG: \url{http://www.cimaglobal.com/Thought-leadership/Newsletters/Insight-e-magazine/Insight-Archive/Are-you-managing-your-spreadsheet-risk/}} \footnote{¿Cuáles son los riesgos de las hojas de cálculo? \url{http://www.cimaglobal.com/Thought-leadership/Newsletters/Insight-e-magazine/Insight-Archive/Are-you-managing-your-spreadsheet-risk/}}) requieren de acceso a hojas de cálculo en una configuración móvil \cite{Flood2011}, especialmente en estos contextos de uso:
\begin{itemize}
	\item En el día a día o tareas cotidianas
	\item Demostración de datos a clientes
	\item Debido al uso del correo dentro de 
\end{itemize}